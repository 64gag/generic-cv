\documentclass[11pt,a4paper,sans]{moderncv}	% opciones posibles incluyen tamaño de fuente ('10pt', '11pt' and '12pt'), tamaño de papel ('a4paper', 'letterpaper', 'a5paper', 'legalpaper', 'executivepaper' y 'landscape') y familia de fuentes ('sans' y 'roman')

\moderncvstyle{classic}				% 'casual', 'classic', 'oldstyle' y 'banking'
\moderncvcolor{grey}				% 'blue', 'orange', 'green', 'red', 'purple', 'grey' y 'black'
%\renewcommand{\familydefault}{\sfdefault}
\nopagenumbers{}

\usepackage[utf8]{inputenc}
\usepackage[scale=0.75]{geometry}
\setlength{\hintscolumnwidth}{2.7cm}

\name{Pedro}{Aguiar}
\title{Curriculum Vitae}
\address{Valle de los Olivos 808}{CP. 66350, Santa Catarina}
\phone[fixed]{+52~(81)~8059~2494}
\phone[mobile]{+52~(81)~8280~8124}
\email{paguiar32@gmail.com}
%\homepage{paguiar.net}
\social[github]{paguiar}
\photo[64pt][0pt]{picture}

%\makeatletter
%\renewcommand*{\bibliographyitemlabel}{\@biblabel{\arabic{enumiv}}}
%\makeatother

%\usepackage{multibib}
%\newcites{book,misc}{{Libros},{Otros}}

\begin{document}
\maketitle
Soy un ingeniero en mecatrónica que, además de haber cumplido y dominado el programa de estudios de mi carrera, también me involucré constantemente en proyectos y actividades paralelas durante la misma; avalando mi emprendedurismo capacidades de autoaprendizaje, aplicación de conocimiento para solucionar problemas, organización, trabajo bajo presión y responsabilidad. Mis áreas de experiencia más desarrolladas son la programación en general (comencé a los 13 años, autoestudio, puesta en práctica en comunidades homebrew), los sistemas embebidos (circuitos digitales, integración y comunicación, control, C), informática en general y Linux (mi sistema operativo de elección desde el año 2006, experiencia con [c]make, gcc, g++, git, vim, ssh, netcat).

% ----------------------------------------------------------------- %
\section{Experiencia Técnica}
\cventry{2014}{Practicante}{\small Laboratoire de Conception et d'Intégration des Systèmes}{}{4.5 meses}
{Estancia en Francia. Sistema embebido de visión para control de un vehículo. Integración y comunicación de cámaras con un STM32 utilizando I2C, DCMI y DMA. Comunicación con un Raspberry mediante SPI. Procesamiento de imágenes acelerado utilizando el GPU del Raspberry y políticas de real-time scheduling de Linux.}

\cventry{2014--2015}{Practicante}{Vehículo Autónomo Supermilla}{}{1 año}
{Participación en el diseño del mecanismo, selección y compra de componentes para la dirección drive-by-wire; integración y programación de microcontroladores (UART, PWM, I2C) y Raspberry en una red CAN; selección de componentes electrónicos y microcontroladores; visión computarizada (modelado y simulación del vehículo y su entorno, programación). Proyecto estudiantil, no implementado.}

\cventry{2013}{Practicante}{Fukushima Lab. Tokyo Institute of Technology}{}{6 meses}
{Trabajo a distancia (videoconferencias). Diseño, desarrollo e implementación de software disftribuido para robots de servicio. Diseño, desarrollo e implementación de software distribuido utilizando middleware (RT-Middleware y ROS, comunicados por websockets) para controlar un robot de servicio (simulado en V-REP). El robot reacciona a señalamientos detectados por una cámara.}

\cventry{2010--2014}{Becario}{Vicerrectoría Académica UDEM}{}{4 años}
{Propuesta e implementación de soluciones a diferentes problemas de manejo de información. Plataforma WLAN de colaboración (HTML, AJAX), plataforma de seguimiento de planeación (PHP, MySQL), recolección y procesamiento automático de inormación de sitios web (javascript), interface entre Excel y formularios HTML (javascript), generación de documentos desde una base de datos (VBA).}

\cventry{2014--2015}{Practicante}{Centro de Innovación en Diseño de Empaque ABRE}{}{8 meses}
{Tareas técnicas como reportes de resultados de proyectos, redacción de documentos técnicos, algunas participaciones en proyectos con empresas, investigaciones/exploraciones tecnológicas, configuración de herramientas y desarrollo de software. Tareas administrativas de interacción con clientes (teléfono e email), elaboración de formatos para documentos y elaboración de presentaciones.}

\cventry{2006--Hoy}{Computólogo/Developer}{Hobby/Autoaprendizaje}{}{8 años}
{Conocimientos de programación, redes, protocolos de internet, servidores, sistemas de información, criptografía, UNIX/Linux, computer architectures. Participación activa en comunidades de homebrew y software de código abierto. Experiencia en programación de bajo nivel de abstracción y uso Linux como mi SO. Aprendo de tecnologías de este tipo regularmente.}

\cventry{2013}{Colaborador}{Investigación Vibraciones Mecánicas}{}{5 meses}
{Algoritmo novedoso desarrollado en MATLAB por el Dr. Santiago Cruz para aproximar soluciones de vibraciones de estructuras utilizando el método de elementos finitos y la transformada de Laplace. Modificaciones al software para hacerlo modular, graficar y simular los resultados dibujando y animando las gráficas y sus estructuras.}

\cventry{2012--2013}{Concursante}{Equipo de la The Freescale Cup}{}{2 años}
{Concurso de diseño de carros seguidores de línea a escala organizado por Freescale Semiconductor y Continental Automotive Mexico. Programación en C de la tarjeta MPC5604B de Freescale Semiconductor, integración de cámara lineal (ADC) y servomotores (PWM).}

% ----------------------------------------------------------------- %
\section{Otras experiencias}
\cventry{2013}{Profesor}{Preparatoria Politécnica de Santa Catarina}{}{1 año}
{Un semestre de servicio social y otro de voluntariado impartiendo clases de Física y Dibujo mecánico asistido por computadora a grupos de un promedio de 40 alumnos cada uno. Los alumnos atendidos fueron de 4to (dibujo mecánico), 5to (Física I) y 6to (Física II) semestre.}

\cventry{2013}{Secretario}{Capítulo estudiantil SAEUDEM}{}{1 año}
{Colaboración en organización de eventos para estudiantes de la comunidad de Ingeniería UDEM como parte del capítulo estudiantil de la Society of Automotive Engineers.}

\cventry{2012--2013\\2008--2010}{Coordinador}{Grupos parroquiales}{}{3 años}
{Actividades formativas y recreativas con niños de alrededor de 10 años de edad que formaban parte del grupo de monaguillos de las parroquias de San Judas Tadeo (Ensenada, Baja California) y San Juan Nepomuceno (Santa Catarina, Nuevo León).}

%\newline{}

% ----------------------------------------------------------------- %
\section{Formación y distinciones académicas}
\cventry{2010--2015}{Ingeniería en Mecatrónica}{Universidad de Monterrey}{}{}
{Graduado con distinción \emph{Cum Laude}, becado durante toda la carrera.}

\cventry{2014--2015}{Becario}{Roberto Rocca Education Program}{}{}
{\href{http://www.robertorocca.org/en/}{Sitio web del programa: http://www.robertorocca.org/en/}}

\cventry{2008--2010}{Técnico en Informática}{CBTis No. 41}{}{}
{Primer lugar en aprovechamiento de informática en mi generación.}

% ----------------------------------------------------------------- %
\section{Idiomas}
\cvitemwithcomment{Español}{Experto}{Lengua materna}
\cvitemwithcomment{Inglés}{Avanzado}{TOEFL ITP score of 620 (11/18/2010)}

% ----------------------------------------------------------------- %
\section{Computación y software}
\cvitem{Programación}{C (experto), C++, javascript, PHP, ensamblador,  shell, VBA, Java, actionscript, ruby.}
\cvitem{MCUs y PLCs}{Microchip, STM32, Freescale, Festo.}
\cvitem{Robótica}{RT-Middleware, ROS, v-rep, gazebo.}
\cvitem{Computer Vision}{OpenCV, Computer Vision Toolbox.}
\cvitem{Documentos}{\LaTeX, GoogleDocs, Microsoft Office.}
\cvitem{Numérico}{MATLAB, GNU Octave, Scilab.}
\cvitem{CAD}{ProEngineer/CREO, Autodesk Inventor, NX.}
\cvitem{Multimedia}{Gimp, Audacity, Blender (básico).}
\cvitem{Web}{HTML, CSS, Flash, SQL, WebGL, apache, AWS.}
\cvitem{Versionado}{git, svn.}
\cvitem{GPU}{OpenGL.}

% ----------------------------------------------------------------- %
\section{Portafolio accesible por internet}

%\cvitem{VAS}{Publicación ``Towards a supermileage autonomous vehicle" del progreso en el proyecto en una conferencia de ASME (IMECE). Trabajo por publicarse este año. \textbf{\href{http://goo.gl/faVR95}{http://goo.gl/faVR95}}}

\cvitem{GitHub}{Se puede encontrar un poco de trabajo sobre robots y sistemas embebidos en mis repositorios de GitHub. \textbf{\href{https://github.com/paguiar?tab=repositories}{https://github.com/paguiar/}}}

\cvitem{WiiTweet}{Cliente (no oficial) de Twitter para el Nintendo Wii. Además del desarrollo de la aplicación yo también hice la documentación que se encuentra en el enlace. Tecnologías: C, C++, XML, JSON, HTTP, HTTPS, SSL, Twitter API. \textbf{\href{http://wiibrew.org/wiki/WiiTweet}{http://wiibrew.org/wiki/WiiTweet}}}

\cvitem{Sitio web Negociando}{Interfaz sencilla a las emisiones anteriores desde dispositivos móviles. Desarrollé y administro el sitio/servidor. Tecnologías involucradas: HTML5, PHP, AJAX, MySQL, apache, AWS. \textbf{\href{http://negociandoudem.net/}{http://negociandoudem.net/}}}

%\nocite{*}
%\bibliographystyle{plain}
%\bibliography{publications}

\end{document}
